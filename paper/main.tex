\documentclass[conference]{IEEEtran}
\usepackage{graphicx}
\usepackage{amsmath}
\usepackage{cite}
\usepackage{url}

\title{Sleep-Inspired Memory Consolidation in Neuromorphic Systems}
\author{\IEEEauthorblockN{Raunak Lakhmani}
\IEEEauthorblockA{Amity University \\ raunak.lakhmani@s.amity.edu}
}

\begin{document}
\maketitle

\begin{abstract}
This paper presents a hybrid neuromorphic framework based on the brain's sleeping activity and how AI can take an insight from Sleep phases of a human brain. This Proposed architecture has "sleep cycles" just like humans which can switch between active learning periods and rest periods where it recalls what it has learned. We tested this on standard AI training tasks which further might result in AI learning effectively over long periods like humans do.
\end{abstract}

\section{Introduction}
[Motivation + Problem]

\section{Related Work}
[Sleep in neuroscience, continual learning in AI, neuromorphic frameworks]

\section{Methodology}
[Architecture + Sleep simulation (replay, weight tuning)]

\section{Experiments}
[Split MNIST / continual tasks]

\section{Results}
[Tables, graphs, forgetting metric]

\section{Discussion}
[Biological plausibility, future improvements]

\section{Conclusion}
[Summary and impact]

\bibliographystyle{IEEEtran}
\bibliography{bib}

\end{document}
